\documentclass[aspectratio=169,14pt,usenames,dvipsnames]{beamer}

\usepackage[utf8]{inputenc}
\usepackage{fontspec}
\usepackage{enumitem}
\usepackage{calc}

\usepackage{datetime}
\newcommand\builddate{%
   \ifcase \month%
        \or Janeiro%
        \or Fevereiro%
        \or Março%
        \or Abril%
        \or Maio%
        \or Junho%
        \or Julho%
        \or Agosto%
        \or Setembro%
        \or Outubro%
        \or Novembro%
        \or Dezembro%
    \fi\space\number\year%
}

\newcommand{\loadtheme}[1]{%
    \input{themes/#1}%
}
\newcommand{\presentationlanguage}[1]{%
    \usepackage[#1]{babel}%
}

\newcommand{\usecodingsamples}[1]{%
    \usepackage{listings}%
    \input{listings/#1}%
}

% Configura a apresentação para ser executada em tela cheia.
\newcommand{\setfullscreen}{\hypersetup{pdfpagemode=FullScreen}}

% Hide beamer navigation simbols
\beamertemplatenavigationsymbolsempty

%
% Standard frames
%

% coverframe
\newcommand{\coverframe}{%
    \begin{frame} %
        \titlepage %
    \end{frame} %
}

% finalframe{email}
\newcommand{\finalframe}[2][Thank you!]{%
    \begin{frame}%
        \begin{flushright}%
            \huge \textbf{#1}%
            \vfill%
            \large \textbf{#2}%
        \end{flushright}%
    \end{frame}%
}

% bigtitle{title}
\newcommand{\bigtitle}[1]{%
    \begin{frame}%
        \begin{center}%
            \Huge {#1}%
        \end{center}%
    \end{frame}%
}

% citation{cite}{author}
\renewcommand{\citation}[2]{%
    \begin{frame}%
        \begin{center}%
            \vspace{1cm}
            \large \textit{"#1"}\\%
            \vspace{1cm}
            \footnotesize {#2}%
        \end{center}%
    \end{frame}%
}

% bigimage{file}
\newcommand{\bigimage}[2][1.0]{%
    {%
        \usebackgroundtemplate{}%
        \begin{frame}%
            {%
            \makebox[\textwidth][c]{%
              \includegraphics[height=#1\paperheight, width=#1\paperwidth,%
                               keepaspectratio]{#2}%
              }%
            }%
        \end{frame}%
    }%
}


\usecodingsamples{python}

\loadtheme{apple_keynote_black}

\title{Desenvolvendo Jogos com pygame}
\subtitle{}
\author{Rafael Guterres Jeffman}
\institute{}
\date{2019}

\begin{document}

%01
\coverframe

%02
\begin{frame}
    \frametitle{Desenvolver Jogos}

    \begin{columns}
        \begin{column}{0.5\textwidth}
            \begin{itemize}
                \item É divertido.
                \item Tu sempre quis fazer.
                \item Foi a primeira coisa que tu fez com algo que parecia
                um computador.
            \end{itemize}
        \end{column}
        \begin{column}{0.5\textwidth}
            \begin{itemize}
                \item Não precisa ser difícil.
                \item Não é fácil.
                \item Tu quer mostrar pra todo mundo que tu consegue
                desenvolver um jogo.
            \end{itemize}
        \end{column}
    \end{columns}
    \begin{center}
        \item {\Large \textbf{É muito divertido!}}
    \end{center}
\end{frame}

%03
\begin{frame}
    \frametitle{Por que Python?}

    \begin{itemize}
        \item É divertido.
        \item Permite que a preocupação seja o problema.
        \item Faz com que tu aprenda uma linguagem que está sendo muito.
        utilizada.
    \end{itemize}
\end{frame}

%04
\begin{frame}
    \frametitle{pygame}

    \begin{itemize}
        \item Não é um engine de jogos.
        \item É uma biblioteca de componentes.
        \item É multi-plataforma (SDL).
        \item Retira as parada chata da programação de jogos.
    \end{itemize}
\end{frame}

%05
\begin{frame}[fragile]
    \frametitle{Hello World}

    \begin{python}
        import pygame
        pygame.init()
        # inicia tela
        screen = pygame.display.set_mode((320,200))
        pygame.display.set_caption("Hello World!")
        # loop principal
        running = True
        while running:
            # trata eventos
            for event in pygame.event.get():
                if event.type == pygame.QUIT:
                    running = False
            # atualiza objetos
            # desenha objetos
            # pygame usa double buffer!
            pygame.display.update()
    \end{python}
\end{frame}

%06
\bigimage{images/moodboard.png}

%07
\begin{frame}
    \frametitle{Every saga has a beginning!}
    \hfill
    \begin{center}
    \large \textbf{Durante um teste de rotina, a nave Genesis é trasportada
    através de um \textit{wormhole} para o quadrante \textit{gamma} da galáxia,
    e precisa sobreviar à Guerra do Infinito.}

    \hfill
    \large\textbf{O que era um dia de testes virou uma luta pela sobrevivência.}
    \end{center}
    \hfill
\end{frame}

%08
\begin{frame}[fragile]
    \frametitle{A janela da aplicação}

    \begin{itemize}
        \item Aplipações pygame podem usar o modo janela ou \textit{fullscreen}.
        \item No modo \textit{fullscreen}, o tamanho da janela é a sua resolução.
    \end{itemize}
    \begin{python}
    width, height = size = (800, 600)
    flags = pygame.FULLSCREEN | pygame.HWSURFACE | pygame.DOUBLEBUF
    screen = pygame.display.set_mode(size, flags)
    \end{python}
\end{frame}

%09
\begin{frame}[fragile]
    \frametitle{Desenhando na tela}
    \begin{itemize}
        \item A estrutura criada pela função \texttt{display.set\_mode} é uma
        superfície, que utilizamos para desenhar na tela.
        \item Esta estrutura pode ser utilizada com o módulo \texttt{pygame.draw}.
    \end{itemize}
    \hfill
    \begin{python}
        python.draw.circle(screen, red, (100,100), 50)
        python.draw.polygon(screen, white, point_list)
        python.draw.rect(screen, white, (x, y, rect_w, rect_h))
    \end{python}
\end{frame}

%10
\begin{frame}
    \frametitle{Um campo de estrelas}

    \begin{itemize}
        \item Um campo de estrelas com três planos pode ser criado com círculos
        que se movem com velocidades diferentes.
        \item \textit{List comprehensions} são muito úteis para isso.
    \end{itemize}
\end{frame}

%11
\begin{frame}[fragile]
    \frametitle{Um campo de estrelas - Criação}

    \begin{python}
    def create_star(x):
        y = randint(0, height)
        speed = choice([1, 2, 3])
        magnitude = choice([1, 2, 3])
        color = (coice(100, 200, 250),) * 3
        return (x, y, speed, magnitude, color)

    stars = [create_star(randint(0, width)) for star in range(count)]
    \end{python}
\end{frame}

%12
\begin{frame}[fragile]
    \frametitle{Um campo de estrelas - Movimentação}

    \begin{python}
    stars = [[x - speed, y, speed, mag, color]
             if x - speed > 0
             else create_star(width)
             for x, y, speed, mag, color in stars]
    \end{python}
\end{frame}

%13
\bigimage{images/starfield.png}

%14
\begin{frame}
    \frametitle{Sprites}

    \begin{itemize}
        \item A first level items.
        \item Another first level items.
    \end{itemize}
\end{frame}

%12
\begin{frame}
    \frametitle{Respondendo a eventos}

    \begin{itemize}
        \item A first level items.
        \item Another first level items.
    \end{itemize}
\end{frame}

%13
\begin{frame}
    \frametitle{Sprites com animação}

    \begin{itemize}
        \item A first level items.
        \item Another first level items.
    \end{itemize}
\end{frame}

%14
\begin{frame}
    \frametitle{Objetos em Escala}

    \begin{itemize}
        \item A first level items.
        \item Another first level items.
    \end{itemize}
\end{frame}

%15
\begin{frame}
    \frametitle{Efeitos especiais}

    \begin{itemize}
        \item A first level items.
        \item Another first level items.
    \end{itemize}
\end{frame}

%16
\begin{frame}
    \frametitle{Disparos}

    \begin{itemize}
        \item A first level items.
        \item Another first level items.
    \end{itemize}
\end{frame}

%17
\begin{frame}
    \frametitle{Tratamento de Colisões}

    \begin{itemize}
        \item A first level items.
        \item Another first level items.
    \end{itemize}
\end{frame}

%18
\begin{frame}
    \frametitle{Vida e morte no espaço}

    \begin{itemize}
        \item A first level items.
        \item Another first level items.
    \end{itemize}
\end{frame}

%19
\begin{frame}
    \frametitle{Escrevendo na tela}

    \begin{itemize}
        \item A first level items.
        \item Another first level items.
    \end{itemize}
\end{frame}

%19
\begin{frame}
    \frametitle{Um pouco de som}

    \begin{itemize}
        \item A first level items.
        \item Another first level items.
    \end{itemize}
\end{frame}

%20
\begin{frame}
    \frametitle{Quão produtivo é o pygame?}

    \begin{itemize}
        \item A first level items.
        \item Another first level items.
    \end{itemize}
\end{frame}

%21
\bigtext{Essa palestra e esse jogo foram (re)feitos em 48h.}

%22
\bigtext{E eu dormi... ;-)}

%23
\bigtext{\url{https://pygame.org}}

%24
\finalframe[Obrigado!]{
    \url{mailto:rafasgj@gmail.com}\\
    \url{https://rafaeljeffman.com/tchelinux}
}

\end{document}
